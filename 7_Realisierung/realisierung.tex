\chapter{Realisierung}\label{chp:realisierung}
Selbstverständlich sollte Realisierung auch in Ihrer Arbeit abgehandelt werden. Aus der Sicht der Softwaretechnik stellt sie aber nur der kronende Abschluß der Arbeit dar. Hier können die realisierungsspezifischen Probleme (z.B. mit der Implementierungssprache) und das Testkonzept inkl. der protokollierten Testergebnisse dargestellt werden. Wichtige, komplexe oder besonders interessante Systemteile können auch im Programmcode dargestellt werden. Bitte aber Hinweis 12 (Programm-Listings) beachten!


\lstset{language=C++,
				backgroundcolor=\color{light-gray},
				%frame=single,
				tabsize=2,
				numbers=left,
				numbersep=5pt,
				%numberstyle=\color{light-gray},
				basicstyle=\ttfamily\color{black}\small,
				keywordstyle=\color{HKS51}\bfseries,
				commentstyle=\color{HKS13}\slshape,,
				identifierstyle=\color{black}}
\begin{lstlisting}	

HANDLE WINAPI CreateFile(
  _In_      LPCTSTR lpFileName,
  _In_      DWORD dwDesiredAccess,
  _In_      DWORD dwShareMode,
  _In_opt_  LPSECURITY_ATTRIBUTES lpSecurityAttributes,
  _In_      DWORD dwCreationDisposition,
  _In_      DWORD dwFlagsAndAttributes,
  _In_opt_  HANDLE hTemplateFile
);

\end{lstlisting}
