\chapter{Systementwurf}\label{chp:systementwurf}
Auf der Basis des Pflichtenhefts werden aus softwaretechnischer Sicht die Anforderungen an das System spezifiziert. Hierzu gehört minimal eine Beschreibung auf höherem Niveau (Modulebene), eine auf mittlerem Niveau (Struktogramme, Pseudocode oder Spezifikation von Prozeduren (Funktionen) sowie die Beschreibung der für das System essentiellen Datenstrukturen (z.B. als Datenlexikon). Typische Beschreibungen sind die Modulhierarchie (oder Modulgraph), eine Spezifikation aller Module mit ihren Schnittstellen (inklusive Zweck, Ein-/Ausgabe), sowie eine Spezifikation aller in den Modulschnittstellen liegenden Prozeduren und Funktionen.
Bestandteil des Entwurfs sollten nicht nur die jeweiligen Ergebnisse, sondern auch die Beschreibung des Entwicklungsweges (inklusive verworfener Lösungen) sein.