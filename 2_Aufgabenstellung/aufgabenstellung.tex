
\chapter{Aufgabenstellung}\label{chp:aufgabenstellung}
\paragraph{}
Das Tool soll die RS-232 Schnittstelle testen und die Schwächen der vorhandenen Lösungen ergänzen. Die Firma Wincor Nixdorf besitzt derzeit schon ein Tool welches die Druckertestumgebung testet, dies ist aber auf Ruby basierend. Das Programm "`WN Serial COM Port Tester"' soll mit Funktionsaufrufen der WinAPI programmiert werden. Die aktuell vorhandenen Lösungen, wie zum Beispiel das Programm "`AGGsoft Serial Port Tester"' haben eingeschränkte Einstellungen und sind nicht ausführlich genug für die Voraussetzungen der Qualitätssicherung.

\section{Module}
\paragraph{}
Das Programm bietet dem Nutzer folgende Schnittstellen an um es zu bedienen:
\begin{itemize}
\item Windows-GUI(Graphical User Interface) mit allen Einstellmöglichkeiten und Start-Button.
\item Kommandozeilen User Interface mit passenden Einstellmöglichkeiten, wie auch per GUI und Rückgabewerte (Exitcodes).
\item Funktionskern, der aus beiden UI angesprochen wird und die eigentlichen Tests durchführt.
\end{itemize}


\section{Laufzeit}
\paragraph{}
Das Com-Port-Test Programm kann von einem schreibgeschütztem Medium (z.B. Windows-PE-CD) laufen, d.h. alle Schreibfunktionen (Log-Datei) sind per Option abschaltbar. Die Testparameter können in einer Konfigurationsdatei gespeichert und wieder geladen werden. Alle Ereignisse und Meldungen können in einer Log-Datei eingetragen werden. 

Wenn das Programm mit einem Skript ausgeführt wird startet das Com-Port-Test Programm und testet die Ports automatisch. Unter Ports sind die im die BEETLE\footnote{Wincor Nixdorf Kassensystem} Systeme eingebaute COM Ports zu verstehen. Solche Ports sind Legacy oder durch PCI erweiterbare Ports.


\section{Hardware}
\paragraph{}
Folgende Hardware-Kombinationen werden unterstützt (die Reihenfolge hier gibt die Priorität der Implementierung wieder):
\begin{enumerate}
\item Test einer Schnittstelle mit Kurzschlussstecker auf einem System
\item Test zweier Schnittstelle mit Null-MODEM-Kabel (RxD/TxD gekreuzt; RTS/CTS gekreuzt, etc.) auf einem System
\item Test zweier Schnittstelle mit Null-MODEM-Kabel auf unterschiedlichen Systemen! Siehe dazu auch „Synchronisierungsprotokoll“…
\end{enumerate}



\section{Betriebssystem}
\paragraph{}
Das Programm "WN Serial COM Port Tester" ist für folgende Zielbetriebssysteme entwickelt worden:
\begin{itemize}
\item Windows XP SP3 x86, x64; Windows 7 x86, x64; WinPE 2.x (WinXP) / 3.x (Win7)
\item Windows 8 nur ein Test, der aussagt ob es kompatibel ist oder noch etwas zu tun wäre
\end{itemize}
Linux - Distributionen werden vom Programm nicht unterstützt.

\section{Logbuch}
\paragraph{}
Alle Ereignisse und Testschritte können zu Dokumentations- und Analysezwecken in einer Log-Datei gespeichert werden. Auch Fehler, die der Parser in der Test-Datei(.ini) findet, werden in der Log-Datei eingetragen. Der Name der Log-Datei ist für jede Instanz und jedes Testsystem eindeutig zu wählen, wie im folgenden Beispiel zu sehen ist:
\\
\\ \hspace*{10mm}\textit{<Programmname>\_<Computername>\_<Master/Slave>\_<Port>.csv}


Jeder Eintrag bekommt einen Uhrzeit und Datums-Stempel. Eine Zeile sieht dann wie folgt aus:
\\
\\ \hspace*{10mm} \textit{<Datum>; <Uhrzeit>;<Ereignis>;<Wert>;<Kommentar>}


Die ersten beiden  Einträge der Log-Datei:
\\
\\ \hspace*{10mm} \textit{<Datum>; <Uhrzeit>;START;RS232-Port-Nummer; INI-Dateiname als Kommentar}



\section{Programminstanzen}
\paragraph{}
Das Programm kann in mehrfachen Programm-Instanzen gestartet werden, in der Regel läuft bei umfangreichen Tests für jeden zu testenden RS-232-Port eine Programm-Instanz. Das vermeidet aufwändige Thread-Programmierung eines einzigen Programms für alle Ports. Das ist keine feste Vorgabe, der Entwickler kann entscheiden welcher Lösungsweg  als geeignet befindet.

\section{Master / Slave}
\paragraph{}
Jede Programminstanz wird als Master, Slave oder mit beiden Konfigurationen gestartet.


\subsection{Master-Funktion}
\paragraph{}
Im Master-Modus schickt die Schnittstelle die vorgegebenen Texte / Dateien mit den eingestellten Parametern an den Slave. Der Master wertet die dem Test zugeordnete INI-Datei aus, der Slave folgt nur dem Master. Der Master wartet auf das Echo oder auf eine Fehlermeldung und bewertet den Testschritt sobald der Test komplett und fehlerfrei empfangen worden ist. Zugleich bewertet er ob der Test erfolgreich oder fehlerhaft war. In beiden Fällen werden Fehler, Warnungen und Erfolgsmeldungen geloggt.


\subsection{Slave-Funktion}
\paragraph{}
Im Slave-Modus (Programminstanz auf gleichem System oder zweiten BEETLE/PC)werden die Daten komplett empfangen, wertet eventuelle Fehler aus und schickt entweder die Daten als Echo oder eine Fehlermeldung wieder zurück. 

\subsection{Master / Slave Kombination mit Kurzschlussstecker}
\paragraph{}
Im kombinierten Master / Slave-Modus (Kurzschlussstecker) übernimmt eine Instanz des Programms beides, das heißt, sie schickt sich selber Daten und antwortet ebenso. Programm intern laufen aber die beiden Module (Master / Slave) weitestgehend getrennt ab.

\subsection{Master / Slave Ablauf mit „Wobbeln“-Funktion}
\paragraph{}
Die Master-Instanz übernimmt die Steuerung des Ablaufes, der Slave antwortet dem Master. In diesem Testmodus werden durch den Benutzer eingegebene Baudraten(Unter- und Obergrenze) gewobbelt\footnote{Für jede Standardbaudrate die einstellbar ist, inklusive der Unter- und Übergrenze, wird ein Test durchgeführt (aufsteigend)}. Der Wobbeln-Ablauf kann über die GUI oder in der INI-Datei beschrieben werden. Der automatische Test erfolgt dann nur für die dort eingetragenen Parameter, z.B.:

\begin{center}
BAUD = 9600 - 115200
\end{center}


Hier werden die Baudrate von 9600 bis 115200 für den Test berücksichtigt, aber 2400, 300, etc. weiter nicht. Wenn mehrere Parameter flexibel gesetzt / programmiert sind, werden alle Kombinationen getestet. Das kann dann durchaus eine nicht unerhebliche Testzeit beanspruchen.\\

Die Programminstanz öffnet die zugehörige Schnittstelle automatisch mit der ersten von allen der programmierten Optionen (Baudrate, Parität, etc.) und startet dann einen Testdurchlauf mit diesen Einstellungen. Wenn alle Daten fehlerfrei gesendet und danach auch das Echo fehlerfrei empfangen wurden,  wird der Port geschlossen, mit der nächsten Kombination geöffnet und wieder Daten gesendet und empfangen. Wenn Fehler (Parität, Offline, etc.) vom Master erkannt oder vom Slave gesendet wurden, dann wird der letzte Schritt wiederholt. Jedes Lese- oder Schreibvorgang wird fünf mal versucht. Wenn der aktuelle Test schon die x-te Fehler-Wiederholung war, dann stoppt das Programm und zeigt den Fehler an. Dieser Fehler-Zustand darf durch das Programm nicht verändert werden. Der Tester / Entwickler muss die Fehler analysieren.

\subsection{Synchronisierungsprotokoll}
\paragraph{}
Für den Wobbelmodus ist ein minimales Protokoll zur Synchronisation nötig. Wenn eine Programminstanz als Master gestartet wird, erhält sie die nötigen Schnittstellenparameter per GUI oder aus der Konfigurationsdatei. Sie sendet diese in From eines Konfigurationspakets dem Slave. Solche Konfigurationspakete werden immer mit den Standardschnittstellenparametern (96,o,8,1) gesendet. Der Slave synchronisiert sich dann, d.h. programmiert -per WIN-API- seinen zu testenden Port mit den empfangenen Parametern und wartet auf Nutzdaten.\\

Wenn der erste Test erfolgreich war (Master hat gesendet und der Slave hat ein Echo geschickt), dann schickt der Master an den Slave die Parameter des nächsten Testschritts. Das ist nötig, weil im Fehlerfall (Echo fehlerhaft empfangen) der Slave nicht mehr synchron wäre, er weiß nur ob seine Daten fehlerfrei gesendet wurden, aber nicht ob der Empfang richtig angekommen ist. Der Slave kann so einfach nur mit Standard-Parametern gestartet werden und folgt dann dem Master.


\section{Teststruktur}
\paragraph{}
Gefordert ist eine einfache Trennung von Testablauf / Testoptionen  und  Testobjekt(RS-232-Port). Die Teststeuerung wird daher sowohl im Programm selber, als auch über die Kommandozeile implementiert. Der Testablauf für jeden Port selber ist in der zugehörigen Konfigurationsdatei vollständig, aber unabhängig vom zu testenden Port (RS-232-Port / V.24-Schnittstelle), beschrieben. Es können mehrere Konfigurationsdateien auf der Kommandozeile übergeben werden. Die Schnittstellenummer wird über die Kommandozeile festgelegt. So kann der gleiche Testablauf (INI-Datei) für jeden Port einzeln oder parallel durchgeführt werden.


\section{Benutzerschnittstelle}

\subsection{GUI}
\paragraph{}
In der GUI sind alle Programm- und Schnittstellenparameter einstellbar und können jederzeit aus der GUI in eine Konfigurationsdatei gespeichert werden. Das Windows-Fenster bietet also implizit auch die Funktion eines Konfigurationseditors mit eingebauter Syntaxüberprüfung.


\subsection{Kommandozeile}
\paragraph{}
Wenn die Kommandozeile keine Parameter bekommt, wird automatisch die GUI dargestellt. Die Kommandozeile nimmt folgende Optionen entgegen:
\begin{itemize}
\item -f INI\_Datei
\item -f INI\_Datei /Port
\end{itemize} 
In der INI Datei ist der Testvorgang beschrieben. Wenn kein "/Port"' angegeben wurde, dann wird der Port in der INI Datei getestet. So sieht ein Programmaufruf aus:
\begin{itemize}
\item SerialPortTester.exe KOMPLETT.INI
\item SerialPortTester.exe KOMPLETT.INI /COM3 
\end{itemize} 

\section{Skriptfähigkeit}
\paragraph{}
Per Kommandozeile gestartet, ist das Programm BATCH-/Script-fähig, d.h. es startet ohne weitere Benutzereingaben, loggt alle Ereignisse und auch Hinweise -zur Analyse nach dem Test- in die zugehörige Datei und beendet sich ebenso selbstständig. Fehler, Warnungen und Hinweise werden auf der Konsole ausgegeben. Als Option kann es auch im Hintergrund ohne jede Anzeige laufen.


\section{Programmierung der Testparameter}
\paragraph{}
Die Testparameter können direkt per GUI oder per INI-Datei eingestellt werden. Die aktuelle Testkonfiguration und der Ablauf kann in einer INI-Datei gespeichert und wieder geladen werden. Eine mit GUI erstellte INI-Datei kann über die Kommandozeile geladen werden. Diese INI-Datei hat die Windows-übliche Struktur mit „[“ und „]“ und ist per Notepad (oder anderen Editor) editierbar.
\\
Sie enthält alle Informationen zur Steuerung der Programminstanz, dass heißt alle einstellbaren Schnittstellenparameter inklusive Wobbelnoption (von-bis):
\begin{tabbing}
\hspace*{10mm}Baudrate = 1200-9600; \=Stopbbits = 1; Datenbits = 8; RTS\_CTS=yes; etc.
\\
\hspace*{10mm}Baudrate = min-max; \>Stopbbits = min-max; etc.
\end{tabbing}

Für MIN/MAX siehe Schnittstellenoptionen. Timeout-Werte (für Echo-Antworten auf Datenpakete) sollten keine festen Zeiten programmiert sein, das Programm  berechnet die Wartezeiten automatisch aus der aktuellen Baudrate. Verzögerungszeiten zwischen zwei Datenblöcken/Testschritten inklusive Zufallswert ist wünschenswert.


\subsection{Schnittstellenoptionen zu Test}
\paragraph{}
Folgende Werte sollen für den Benutzer einstellbar sein:

\begin{itemize}
\item Baudrate
\item Parität
\item Datenbits
\item Stopbits
\item Flusssteuerung
\end{itemize}

Das Programm kann hier mit Variablen MIN und MAX umgehen. Wenn beispielsweise die Zeile „BAUDRATE = MAX“ oder „BAUDRATE = MIN“  programmiert wurde,  sucht das Programm die maximale bzw. minimale Baudrate, die angegebene RS-232-Port unterstützt und nutzt sie für den Test. Diese Werte werden aus „dem System“ –beispielsweise Registry oder  MSPORTS.DLL direkt ausgelesen. Manuell können diese MIN/MAX-Werte über den Gerätemanager ermittelt werden. Wenn keine Parameter übergeben wurden, startet das Programm mit diesen Standardparametern: 9600,odd,8,1,RTS/CTS (von der Wincor Nixdorf Anzeige BA66).

\paragraph{}
Die Wobbelnoption wird in Form einer VON-BIS-Zeile implementiert:

\begin{tabbing}
\hspace*{10mm} BAUDRATE=300;2400;9600;115200 \= diese vier Werte werden getestet
\\
\hspace*{10mm} BAUDRATE=MIN-MAX \>alle möglichen Werte
\\
\hspace*{10mm} BAUDRATE=300-9600;115200 \>alle Werte von 300 bis 9600 und 115200 im Test
\end{tabbing}


\subsection{Sendedaten aus Datei}
\paragraph{}
Als Testdaten werden feste Texte (in der Regel ASCII Zeichen) oder auch Pattern aus einer Datei gesendet. Die Dateipfade / Sendetexte sind als Testparameter in der GUI oder Test-Datei) einstellbar. Wenn keine Sendedaten übergeben wurde, dann sendet das Programm zufällige, aber auf einem Protokollanalysator lesbare, ASCII-Zeichen: 0x20 ... 0x7F .


\section{Fehlererkennung und Behandlung}
\paragraph{}
Folgende mögliche Fehlerfälle muss ein solches Tool erkennen:
\begin{itemize}
\item Kabel-/Stecker-Fehler („Wackelkontakt“ durch schlechten mechanischen Kontakt Stecker/Buchse) 
\item Elektrische Fehler, die sich bei Übertragung über langes Null-MODEM-Kabel im Form von Paritäts-, Rahmenfehlern oder ähnliches zeigen
\item Dauerhaft gezogene Kabel / Kurzschlussstecker
\item Windows-interne Ressourcenprobleme (Shared Interrupts)
\end{itemize} 

Bei Erkennung eines Übertragungsfehlers wird dieser in die Logdatei eingetragen und / oder auf der Konsole angezeigt und danach der letzte Block wiederholt. Nach der x-ten erfolglosen Wiederholung (Wert aus Test-Datei) erfolgt Testabbruch

%%%%%%%%%%%%%%%%%%%%%%%%%%%%%%%%%%%%%%%%%%%%%%%%%%%%%%%%%%%%
\newpage
%%%%%%%%%%%%%%%%%%%%%%%%%%%%%%%%%%%%%%%%%%%%%%%%%%%%%%%%%%%%

\section{Neue Anforderungen}
\subsection{Logbuch}
\paragraph{}
Der Name der Logdatei soll um eine ID-Nummer erweitert werden. Wenn eine Logdatei auf dem Zielsystem schon vorhanden ist, wird diese Nummer hochgezählt.\\

\hspace*{10mm}\textit{<Programmname>\_<Computername>\_<Master/Slave>\_<Port>\_<Nummer>.ini}

Wenn der Test direkt über die GUI konfiguriert wurde, dann werden diese Werte in einer INI-Datei vor dem Teststart gesichert und diese in die Logdatei eingetragen.

\subsection{Synchronisierungsprotokoll}
\paragraph{}
Das erste solcher Konfigurationspakete wird immer mit den Standardschnittstellenparametern (96,o,8,1) gesendet. Ein solches Paket beginnt mit einem Fluchtzeichen (in ASCII wäre das ESCAPE 0x1B) und wird vom Slave per Bestätigungszeichen (in ASCII ACK) beantwortet.


\subsection{Anhalten eines Tests}
\paragraph{}
Möchte der Benutzer ein Test vorzeitig beenden, gibt es eine Schaltfläche in der GUI die den Test anhält. Wird per Kommandozeile das Programm ausgeführt, kann der Benutzer durch drücken der "`ESC"' Taste ein Test stoppen.