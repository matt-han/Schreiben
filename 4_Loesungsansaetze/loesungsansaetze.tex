\chapter{Lösungsansätze}\label{chp:loesungsansaetze}

\section{Fehlererkennung}
\paragraph{}
Das Ziel des COM-Port-Test-Programm ist Fehler in der Systemhardware, beziehungsweise in der Kommunikation zwischen den Schnittstellen zu erkennen. Damit dieses Ziel erreicht wird, muss die Ermittlung eines Fehlers äußerst genau geschehen. So muss jeder Funktionsaufruf und Kommunikationsvorgang immer abgefragt werden. Bei jeder Abfrage müssen die Fehler ausgewertet und gemeldet werden. Nur so erfüllt das Testprogramm seinen Zweck. Die Herausforderung wird darin bestehen, die Fehler genau zu definieren und sie so zu melden, dass der Tester die Ursache sofort erkennen kann.

\section{Master Slave Modus zwischen zwei Systemen}
\paragraph{}
Die Synchronisierung zwischen zwei verschiedenen BEETLE Systemen wird eine große Herausforderung darstellen. Dafür wird ein kleines Protokoll oder Handshake definiert. In diesem Vorgang muss der Master wissen, dass der Slave fertig eingestellt und bereit für ein Datenverkehr ist. Dabei müssen beide Schnittstellen die richtige Einstellungsparameter (Baudrate, Paritätbits, Stopbits, Datenbits, etc.) besitzen und konfigurieren.\\


Um das Protokoll oder Handshake zu definieren gibt es verschiedene Möglichkeiten. Zusätzlich müssen zuerst die Timeouts beachtet werden. Der Master darf keine Information oder Übertragungstexte senden, bevor der Slave nicht in der Lage ist, Daten zu empfangen. Wäre der Slave nicht bereit sein, würde der Master ein Lese-Timeout melden, weil er Daten verschickt hat, aber der Empfänger keine Daten zurück versendet. Im Gegensatz, darf der Slave nicht Daten erwarten, so lange der Master nicht in der Lage ist, Daten zu verschicken.\\


Als Grundidee, um dieses Problem zu lösen, werde ich Synchronisierungszeiten und Vorgänge einbauen, in dem der Master für eine definierte Zeit auf den Slave wartet. Wenn diese Zeit abgelaufen ist, wird dem Benutzer gemeldet, dass es Probleme mit der Synchronisierung beider Systeme gab und dass kein Test stattfinden wird. Ein großer Nachteil dieser Lösung ist, dass durch das Abfragen auf Bereitschaft ein Pollingverfahren\footnote{Das zyklische Abfragen eines Status} eingesetzt werden muss. \\


Im besten Fall, wenn keine Fehler bei der Kommunikation auftreten, wird es nicht so nötig sein, sich ständig zu Synchronisieren. Im Fehlerfall muss nach jedem Test eine Synchronisierung stattfinden. Empfängt der Slave verfälschte Informationen und meldet einen Fehler, wurde er sich sofort auf den nächsten Test vorbereiten. Das würde den gesamten Testvorgang verfälschen. Es werden weitere solche Fälle auftreten, auf die unterschiedlich  reagiert werden muss.\\


Um das Pollingverfahren zu vermeiden, kann die Synchronisierung auf Basis von Events \footnote{http://msdn.microsoft.com/en-us/library/aa450667.aspx; 30.08.2013} basieren. Dabei erzeugt die Schnittstelle bei der Ankunft von einem Byte Interrupts. So kann der Slave reagieren, erst wenn der Master tatsächlich etwas verschickt hat. Um die Interrupts abzufragen, muss die Kommunikation Events von Windows abfragen. Wenn das System ein Interrupt empfängt, wird dieses im Programm als Event weitergeleitet.\\


Auf den ersten Blick wird das vollkommene Umgehen eines Pollingverfahrens schwer sein. Die Lösung wird sich durch Tests während der Entwicklung des "WN Serial COM Port Tester" ergeben. Je nach Testergebnis werde ich unterschiedliche Lösungen anwenden.



\section{C++ und die GUI}\label{C++GUILoesung}
\paragraph{}
Die graphische Oberflächen mit Anwendung von der Windows API besteht aus "`C"' Code, und muss eine statische Fensterprozedur besitzen. Das heißt, es können beliebig viele Benutzeroberflächen kreiert werden, aber es gibt nur eine Prozedur, die die Nachrichten bearbeitet. Durch die Objektorientierung unter "`C++"' ersteht ein Konflikt wegen der statischen Fensterprozedur. Um eine statische Methode zu definieren, müssen auch statische Variablen erzeugt werden. So werden diese Variablen nur einmal erzeugt, unabhängig davon, wie viele Fenster erstellt werden. Wenn die Methoden und die Variablen nicht statisch wären, würden bei der Ersetllung von mehreren Fenstern die versendeten Nachrichten durcheinander kommen.\\


Um dieses Problem zu umgehen wird mit Hilfe von einer \textit{Template Class} die Oberfläche definiert. Eine Klasse wird definiert, in der die allgemeinen Eigenschaften eines Fensters gesetzt werden und die Nachrichten von jedem Fenster richtig umgeleitet werden. Dieses geschieht indem die benötigte statische Fensterprozedur in dem Template definiert wird. Die allerersten Nachrichten, die ein Programm erhält, um eine Oberfläche zu erzeugen, ist die "`WM\_NCREATE"'. Die Fensterprozedur fragt nach dieser Nachricht und wenn die Nachricht ankommt, wird durch einen Zeiger der Fensterhandle richtig gesetzt und an die richtige Prozedur weitergeleitet. Da diese Fensterprozedur statisch ist, egal wie viele Oberflächen kreiert werden, wird es nur einmal diese Fensterprozedur geben.\\


Die Fensterklasse erbt von der \textit{Template Class} und kann eine eigene Fensterprozedur definieren. Die definierte Fensterprozedur in der Fensterklasse wird durch das Template die Nachrichten weiterleiten. Dadurch werde ich auch in der Lage sein, meine eigene Fenster so zu gestalten wie die Voraussetzungen es definieren, und falls es nötig ist, mehrere Fenster zu erzeugen.