\chapter{Testbeschreibung}\label{chp:Testbeschreibung}
\paragraph{}
Als letzter Schritt für die Beendigung der Entwicklung des Testprogrammes müssen Test für die Funktionseinheiten und die Benutzerschnittstelle durchgeführt werden. 


\section{Benutzerschnittstelle}
\paragraph{}
In den Tests für die Benutzerschnittstellen müssen alle Parameter getestet werden, in die der Benutzer einen Fehler eingeben kann. Im Testprogramm gibt es einen Abschnitt wo alle Parameter überprüft werden. Folgende Testfälle (siehe Tabelle) müssen von diesem Programmabschnitt abgefangen werden und dem Benutzer gemeldet werden, welche Eingaben fehlerhaft waren. Diese Fehleingaben werden mit einer passende Fehlermeldung dem Benutzer ausgegeben. Folgende Teststrategien werden ausgeübt:

\begin{tabular}{lll}

&\textbf{Positivtest:} &erwartete Eingabe für den jeweiligen Parameter (bei einer Zahl wird eine gültiger\\
& & numerischer Wert eingegeben)\\

&\textbf{Negativtest:} &unerwartete Eingabe für den jeweiligen Parameter (bei einer Zahl wird ein Text\\
& & eingegeben)\\

&\textbf{Grenzwerttest:} &gültige und ungültige Grenzwerte\\
\end{tabular}

Für die Dokumentation der Tests werden die Positivtests nicht berücksichtigt. 


\subsection{INI-Datei}
\paragraph{}
In einer INI-Datei müssen alle gelesenen Parameter auf plausible Werte überprüft werden. Eine Liste mit den Bedeutungen der Errorcodes ist im Anhang \ref{Errorcodes} nach zu schlagen.
\begin{longtable}{||l|p{3cm}|p{5cm}|c||}
 
%  Als Kopfzeile & der anderen Seiten & wird dies gesetzt &\\ \hline
%  \endhead
%  Die Fu"szeilen & f"ur & alle Seiten & \\ \hline\hline
% \endfoot
%  Nur die letzte & Fu"szeile ist etwas & BESONDERES &\\ \hline\hline
%  \endlastfoot


\hline
Testfall & Fehler & Kommentar & Errorcode \\ \hline\hline
\endfirsthead
\hline
Testfall & Fehler & Kommentar & Errorcode \\ \hline\hline
\endhead

COM0   & falscher COM Port & kein gültiger Wert, Datei wird als Leer interpretiert & -12\\ \hline
COM1   & falscher COM Port & nicht im System vorhanden                             & -6 \\ \hline
COM256 & falscher COM Port & nicht im System vorhanden                             & -6 \\ \hline
COM257 & falscher COM Port & kein gültiger Wert, Datei wird als Leer interpretiert & -12\\ \hline\hline

baudrate=Min-Max     & falscher Parameter        & MIN-MAX                      & -17 \\ \hline
baudrate=-MAX        & unvollständiger Parameter & fehlt eine minimale Baudrate & -17 \\ \hline
baudrate=MIN-        & unvollständiger Parameter & fehlt eine maximale Baudrate & -17 \\ \hline
baudrate=-9600-115200& negative Baudrate         & ungültige Eingabe wird nicht interpretiert &  -15\\ \hline

parity=Min-Max & falscher Parameter & MIN-MAX         & -9\\ \hline
parity=1       & falscher Parameter & nur Text & -9\\ \hline
parity=ODD     & falscher Parameter & odd, even, none & -9\\ \hline\hline

databits=6    & falscher Parameter & 7, 8       & -9\\ \hline
databits=9    & falscher Parameter & 7, 8       & -9\\ \hline
databits=text & falscher Parameter & nur Zahlen & -9\\ \hline\hline

protocol=0    & nur Text             & none, hardware, XON/XOFF & -9\\ \hline
protocol=NONE & falscher Parameter & none, hardware, XON/XOFF & -9\\ \hline\hline

stopbits=text & nur Zahlen           & 1, 2 & -9\\ \hline
stopbits=0    & falscher Parameter & 1, 2 & -9\\ \hline
stopbits=3    & falscher Parameter & 1, 2 & -9\\ \hline\hline

transfertextmode=sometext & falscher Parameter & default, text, file   & -9\\ \hline
transfertextmode=default  & & &\\
transfertext=teststring   & unnötiger Parameter  & wird nicht betrachtet & 0 \\ \hline

transfertextmode=file & & &\\
transfertext                  & nicht definiert      & Dateipfad muss angegeben werden     & -9\\ \cline{2-4}
transfertext=                 & leerer Parameter     & Parameter muss angegeben werden     & -9\\ \cline{2-4}
transfertext=C:\textbackslash & ungültiger Dateipfad & gültiger Pfad muss angegeben werden & -13\\ \hline\hline

transfertextmode=text & & &\\
transfertext                  & nicht definiert      & Übertragungstext muss angegeben werden & -9\\ \cline{2-4}
transfertext=                 & leerer Parameter     & Parameter muss angegeben werden        & -9\\ \hline\hline

logger=keine & falscher Parameter & true, false & -9\\ \hline
logger=123   & falscher Parameter & true, false & -9\\ \hline\hline

stoponfirsterror=nein & falscher Parameter & true, false & -9\\ \hline
stoponfirsterror=321  & falscher Parameter & true, false & -9\\ \hline\hline

repeater=-1   & falscher Parameter & nur Zahlen $\geq$ 0 & -9\\ \hline
repeater=text & falscher Parameter & nur Zahlen $\geq$ 0 & -9\\ \hline\hline

\end{longtable}


Bei einer Testdatei mit mehreren Sektionen, werden zuerst alle Eingaben gelesen und geprüft. Wird ein Fehler gefunden, wird dieser dem Benutzer gemeldet. Sind ungültige Sektionen wie [COM0] oder [COM257] in der Testdatei eingetragen, werden diese vom Testprogramm nicht betrachtet. Alle anderen gültigen Eingaben werden geprüft und getestet.


\subsection{Kommandozeile}
\paragraph{}
In der Kommandozeile kann der Benutzer eine vollständige Testdatei angeben oder eine Testdatei wo die Sektion als [COM] definiert ist, plus die Angabe welcher Port getestet werden soll.

\hspace*{10mm}\textit{SerialPortTester.exe C:\textbackslash Testdateien\textbackslash Testdatei1.ini}\\
\hspace*{10mm}\textit{SerialPortTester.exe C:\textbackslash Testdateien\textbackslash Testdatei2.ini /COM1}\\

\begin{longtable}{||l|p{3cm}|p{5cm}|c||}
\hline
Testfall & Fehler & Kommentar & Errorcode \\ \hline\hline
\endfirsthead
\hline
Testfall & Fehler & Kommentar & Errorcode \\ \hline\hline
\endhead

C:\textbackslash Testdateien & falscher Dateipfad & Dateipfad gibt keine Datei an & -12\\ \hline
C:\textbackslash Testdateien\textbackslash Datei & falscher Dateipfad & Dateipfad gibt keine gültige Testdatei an & -12\\ \hline

C:\textbackslash Testdatei.ini /com1 & falscher COM Port & Portbezeichnung ungültig, muss COM1 sein & -1\\ \hline
C:\textbackslash Testdatei.ini /COM0 & falscher COM Port & Portname ungültig & -1\\ \hline
C:\textbackslash Testdatei.ini COM1 & falscher Parameter & Parameter muss mit / beginnen & -1\\ \hline

C:\textbackslash Testdatei.ini /COM1 /COM2& falscher Parameter & zu viele Parameter & -1\\ \hline
/COM0 & falscher Parameter & Dateipfad fehlt &-1 \\ \hline

\end{longtable}


\subsection{Benutzeroberfläche}
\paragraph{}
In der Benutzeroberfläche werden nur gültige Parameter angeboten, außer die Baudraten, der Wiederholungszähler (Repeater), die Übertragungsdatei und der Übertragungstext. Die Baudraten werden für jeden COM Port vom System abgefragt. Durch die Tests wurde festgestellt, dass manche Baudraten angeboten werden, die nicht wirklich unterstützt werden. Der Benutzer kann im Fehlerfall, mit Hilfe der Kommandozeile und den Befehl \textit{mode} sicherstellen, dass die Baudrate tatsächlich nicht unterstützt wird. Bei den anderen Parametern werden Fehler erkannt, da die Überprüfung für die GUI oder Testdatei die gleiche ist.



\section{Funktionseinheiten}
\paragraph{}
Die Funktionstests basieren auf den Funktionen des Testprogrammes. Die verschiedenen Einstellungsmöglichkeiten werden getestet. Durch die Ausführlichkeit des Tests werden nur die durchgefallenen Tests erläutert. Laut den  Testergebnissen gibt es keine Einschränkungen in den Testmodi (Automatic, Wobble und Fixed) und in den Transfermodi \textit{Shorted} und \textit{Double}. Wird der Transfermodus \textit{Master-Slave} in einem System getestet, wurden keine Probleme erkannt. Wird die gleiche Einstellung in zwei verschiedenen Systemen ausgeführt, kommt es öfters zu Synchronisierungsproblemen nach dem ersten Testschritt\footnote{Unter Testschritt ist zu verstehen, dass ein Test aus mehreren Schritten besteht. Für jede mögliche Testkonfiguration wird ein Testschritt definiert. Wird für ein \textit{Fixed-Shorted} Test der Wiederholzähler auf Zehn gesetzt, besteht dieser Test aus Zehn Testschritten}. Wenn beide Schnittstellen im ersten Testschritt sich erfolgreich synchronisieren und den ersten Testschritt erfolgreich abschließen, müssen sich für einen neuen Testschritt nochmal synchronisieren. Bei diesen Vorgang kommt es zu einem Fehler und beide Schnittstellen erkennen sich nicht.