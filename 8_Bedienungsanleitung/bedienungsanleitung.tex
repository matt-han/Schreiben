\chapter{Bedienungsanleitung}
Wie läßt sich das entwickelte System bedienen? Welche Fehlermeldungen existieren? Was sind die Ursachen? Wie muß der Benutzer auf auftretende Bedienungs- oder Systemfehler reagieren? Bedienungsanleitungen sollten auf den Kenntnisstand und Sprachgebrauch des zukünftigen Anwenders zugeschnitten sein. In der Form und Aufmachung können SIe ein Aushängeschild für Ihre Arbeit sein. Übersichtsdarstellungen wir Menübäume, Befehlslisten und/oder Systaxdiagramme erleichtern die Benutzung eines Systems umgemein und gehören heute zum Standard guter Dokumentationen. Neben der Einzeldarstellung von Befehlen und Funktionen können die Aufzeichnung von Beispielsitzungen, Fallbeispielen das Verständnis für das System wesentlich vertiefen.
(Anmerkung: Je nach Art des entwickelten Systems ist es manchmal sinnvoll die Bedienungsanleitung in ein größeres Kapitel Anwenderdokumentation einzubetten.)