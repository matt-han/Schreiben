\chapter{Bedienungsanleitung}\label{chp:bedienungsanleitung}
%Wie läßt sich das entwickelte System bedienen? Welche Fehlermeldungen existieren? Was sind die Ursachen? Wie muß der Benutzer auf auftretende Bedienungs- oder Systemfehler reagieren? Bedienungsanleitungen sollten auf den Kenntnisstand und Sprachgebrauch des zukünftigen Anwenders zugeschnitten sein. In der Form und Aufmachung können SIe ein Aushängeschild für Ihre Arbeit sein. Übersichtsdarstellungen wir Menübäume, Befehlslisten und/oder Systaxdiagramme erleichtern die Benutzung eines Systems umgemein und gehören heute zum Standard guter Dokumentationen. Neben der Einzeldarstellung von Befehlen und Funktionen können die Aufzeichnung von Beispielsitzungen, Fallbeispielen das Verständnis für das System wesentlich vertiefen.

%(Anmerkung: Je nach Art des entwickelten Systems ist es manchmal sinnvoll die Bedienungsanleitung in ein größeres Kapitel Anwenderdokumentation einzubetten.)

\paragraph{}
Das Test Programm kann über die Kommandozeile oder direkt über die Benutzeroberfläche bedient werden. In beiden Fällen muss der Benutzer folgende Parameter angeben, je nach Testmodus, um erfolgreich ein Test zu starten:
\\
\begin{tabular}{llll}
\\ &\textbf{Parameter} & &\textbf{GUI Element}
\\ &COM Port &: &Main Port
\\ &Baudrate &: &Baud rate
\\ &Test Modus &: &Test Mode
\\ &Übertragungsmodus &: &Transfer
\\ &Slave Port &: &Slave Port
\\ &Maximale Baudrate &: &MAX baud rate
\\ &Parität &: &Parity
\\ &Übertragungsprotokoll &: &Protocol
\\ &Stopbits &: &Stopbits
\\ &Datenbits &: &Databits
\\ &Übertrangsdatei &: &Load file to be transfered
\\ &Übertragungstext &: &Send text
\\ &Logger &: &Logger
\\ &Wiederholzähler &: &Repeater
\\ &Start &: &Start
\\ &Schließen &: &Close
\\ &Speichern &: &Save
\\ &Testdatei laden &: &Load test file
\\ &Hilfe &: &Help
\\ &Stop &: &Stop
\end{tabular}


\section{Bedienung über die GUI}
\paragraph{}
Mit der GUI kann der Benutzer ein Test starten, mit bestimmten Parametern oder durch das Laden einer Testdatei. Der Benutzer kann auch eine Testdatei erstellen, in dem er die Parameter in der GUI einstellt und diese speichert.

\section{Bedienung über die Kommandozeile}
\paragraph{}
Über die Kommandozeile kann der Benutzer eine Testdatei und das zu testende Port angeben. Wenn keine Parameter angegeben worden sind, wird die GUI gestartet.

\section{Testmodus}

\subsection{Automatic}
\paragraph{}
Der Automatic Test ist vordefiniert. Für diesen Test muss der Benutzer nur den Port angeben, der er teste will, und welches Übertragungsmodus. Der Test wird immer mit folgenden Parameter gestartet.
\\
\begin{tabular}{llll}
\\ &\textbf{Parameter} & &\textbf{Wert}
\\ &Main Port &: &Von Benutzer wählbar
\\ &Baud rate &: &9600
\\ &Test Mode &: &Automatic
\\ &Transfer &: &Von Benutzer wählbar
\\ &Slave Port &: &Wenn nötig, von Benutzer wählbar
\\ &Parity &: &Odd
\\ &Protocol &: &Hardware
\\ &Stopbits &: &1
\\ &Databits &: &8
\\ &Send text &: &Default, fest kodiert
\\ &Logger &: &Ja
\\ &Repeater &: &Unendlich
\end{tabular}

\subsection{Wobble}
\paragraph{}
In einem Wobble Test kann der Benutzer alle Parameter einstellen. Die wichtigsten Parametern sind die minimale und maximale Baudrate. Diese Werte geben an, mit welchen Baudraten getestet wird.


\subsection{Fixed}
Im Fixed Test muss der Benutzer alle Parameter einstellen. Im diesem Test werden sich die Parameter nicht ändern. Der Benutzer kann nach Fehler mit spezifischen Einstellungen testen.
\paragraph{}


\section{Übertragungsmodus}
\paragraph{}
Jedes Testmodus kann in vier Übertragungsmodus eingestellt werden, die den Test ablauft bestimmen


\subsection{Shorted}
\paragraph{}
Im Shorted Modus wird an der angegebene Schnittstelle eine Kurzschlussstecker angeschlossen. Hier ist der Port Master und Slave gleichseitig. Der Port schickt und empfängt die Daten sofort.


\subsection{Double}
\paragraph{}
Hier werden zwei Ports in einem System getestet. Mittels eines Null-Modem-Kabel wird Port1 and Port2 angeschlossen. Port1 ist der Master und Port2 ist der Slave. So sendet der Master Informationen und der Slave empfängt sie.


\subsection{Master}
\paragraph{}
Der Master Modus involviert zwei verschiedene Systeme. Im System1 schickt der Master Daten und wartet auf eine Antwort. Der Master kennt sein Slave nicht. System1 und System2 werden durch ein Null-Modem-Kabel verwunden. Der Master macht die Fehlerauswertung, so lange keine Lese- oder Schreibfehler im Slave geschehen sind.


\subsection{Slave}
\paragraph{}
Der Slave im System2 wartet auf den Empfang von Daten aus System1. Der Slave kennt sein Master nicht, er empfangt Daten und schickt die gleiche Daten zurück, damit der Master die Fehlerauswertung machen kann.


\section{Schnittstelleneigenschaften}

\paragraph{Baudrate} Die Baudrate mit der getestet werden soll.
\paragraph{Parity} Hier kann der Benutzer zwischen gerade, ungerade oder keine Parität auswählen.
\paragraph{Protocol} Der Benutzer ist in der Lage das Übertragungsprotokoll zu bestimmen. Zur Auswahl steht Xon/Xoff, Hardware oder kein Protokoll.
\paragraph{Stopbits} Mit diesem Parameter werden die Stopbits eingestellt, entweder 1 oder 2 Stopbits.
\paragraph{Databits} Hier wird bestimmt, wie viele Bits ein Zeichen hat, 7 oder 8 Bits pro Zeichen.

\section{Sendedaten}
\paragraph{}
Durch klicken des Buttons "`Load file to send"' kann der Benutzer eine Datei auswählen, diese wird dann beim testen geöffnet und Zeilenweise verschickt. Will der Benutzer ein spezifischer Text senden, muss er dieser unter "`Send text"' eingeben und "`Load text to send"' klicken. Wenn keiner dieser beiden Möglichkeiten benutzt wird, wird ein im Programm fest kodierter Text übertragen.

\section{Logger}
\paragraph{}
Wenn die Checkbox ausgewählt ist, wie eine Textdatei im "`\%Temp\%"' Verzeichnis angelegt und alle Meldung dort geloggt. Die Datei hat immer den Namen des Ports dass getestet wird.

\section{Wiederholzähler}
\paragraph{}
Der Wiederholzähler gibt an, wie oft ein Testschritt wiederholt werden soll. Unter Testschritt ist zu verstehen, jedes Schreibe- und Lesevorgang mit verschiedene einstellbare Parameter.

\section{Buttons}
\paragraph{Start} Beginnt ein Test mit den eingegeben Parametern.
\paragraph{Close} Beendet das Test Tool.
\paragraph{Save} Speichert die ausgewählten Einstellungen in einer Testdatei.
\paragraph{Load test file} Ladet eine vom Benutzer ausgewählte Testdatei.
\paragraph{Help} Zeigt die Hilfe zum Programm
\paragraph{Stop} Stoppt der Testvorgang



\section{Testdatei}
\paragraph{}
In einer Testdatei werden die Testkonfiguration gespeichert. Eine Testdatei kann manuell oder mit Hilfe der GUI erzeugt werden. Die Testdatei ist wie eine Datei aus der Registrierungsdatenbank (Windows Registry)aufgebaut. In "`\[\]"' wird der Port angegeben und darunter die jeweilige Testparametern. Für ein Beispiel bitte siehe Anhang. 

\section{Einfaches Beispiel}
\paragraph{}
Will zum Beispiel ein Benutzer der Port COM1 testen. Dieser soll an COM2 die Wörter "`Hallo Welt"' schicken. Als Baudrate ist 9600 vorgegeben, mit einer geraden Parität, Xon/Xoff Protokoll, sieben Datenbits und 2 Stopbits. Der Test soll fünfmal wiederholt werden und eine Log-Datei soll geschrieben werden.

\paragraph{}
Als erstes muss der Benutzer unter \textit{Main Port} "`COM1"' auswählen. Danach werden die mögliche Baudraten in \textit{Baud rate} aufgelistet. Dort muss der Benutzer 9600 wählen. Da nur eine bestimmte Konfiguration gegeben ist, ohne Variablen, wird auf "`Fixed"' unter \textit{Test Mode} geklickt und als Übertragungsmodus "`Double"' ausgewählt. Unter \textit{Parity} wird "`Even"', \textit{Protocol} wird "`XON/XOFF"', \textit{Stopbits} wird "`2"' und \textit{Databits} wird "`7"' ausgewählt.\\

BILD\\

\paragraph{}
Unter \textit{Send text} soll der Benutzer "`Hallo Welt"' schreiben. Will der Benutzer Escape-Sequenzen schicken, müssen diese als Hexadezimal Werte angegeben werden(zum Beispiel \textbackslash0a oder \textbackslash0d). Danach wird auf \textit{Load text to send} geklickt. Ein Pop-Up Fenster meldet, dass der Text geladen worden ist. Damit der Test fünfmal wiederholt wird, muss in \textit{Repeater} eine fünf anstatt der eins geschrieben werden. Danach ist die GUI konfiguriert und kann auf \textit{Start} geklickt werden.\\

BILD\\

\paragraph{}
Durch ein Pop-Up Fenster wird dem Benutzer gemeldet, wenn der Test fertig ist. Um eine Evaluation des Tests machen zu können muss die Log-Datei ausgewertet werden.

